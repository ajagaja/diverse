% !TeX root = RJwrapper.tex
\title{diveR: An R Package to Keep Diversity Simple}
\author{by Miguel R. Guevara, Dominik Hartmann and Marcelo Mendoza}

\maketitle

\abstract{
This article describes the R package \pkg{diveR} which provides an easy-to-use interface to deal with high volumes of data used in different areas of Social Sciences. Most R packages on Diversity are specialized on the demands and terminology of Life and Biological Sciences. The \pkg{diveR} package presents an useful alternative for social scientists to translate useful concepts from natural science into social science language, and making use of recent large dataset to analyzing the evolution of diversity within social, economic and political systems. The package includes several measures of diversity --like variety, disparity and balance-- and allows to import data in a simple manner.
}


\section{Introduction}
Diversity measures are used in many different fields of science. For example ecologists calculate the diversity of species \citep{forey_systematics_1994}, sociologists the community structure and diversity \citep{haughton_economic_1995}, economists the diversification of exports or financial assets \citep{hidalgo_product_2007}, scientometrists the diversity of research portfolios \citep{rafols_knowledge_2014}, or computer scientists the diversity of ensembles of algorithms \citep{kuncheva_measures_2003}. While diversity measurement has been a natural topic in biology, the rise of complexity research and big data has also implied new opportunities for researchers from social sciences to understand the evolution of diversity in social, economic and political systems or the interaction between social and natural systems

A constituent property of complex systems is the diversity of a system’s elements and interactions. Here we focus on the measurement of diversity of different elements in a complex system, other packages (like \CRANpkg{igraph}) allow exploring in more detail the patterns of interactions and network structure. 

Different fields of science created a set of specialized packages on diversity, exploring for instance  Biodiversity, \CRANpkg{vegan} and \CRANpkg{BiodiversityR}; social distances \CRANpkg{Blaunet}, genetics \CRANpkg{diveRsity}, biological systems \CRANpkg{divo}, species and communities \CRANpkg{entropart}, functional ecology \CRANpkg{FD}, species complexity \CRANpkg{hierDiversity}, bootstrapping diversity indices \CRANpkg{simboot}, disparity of phylogenetic trees \href{https://github.com/thibautjombart/treescape}{treescape} or phylogenetic patterns \CRANpkg{SYNCSA}. However, each discipline uses its particular language and terminology.

The differences in language hampers the mutual understanding and adoption of existing packages, measures and knowledge from different disciplines. As mentioned above, the main packages to work with diversity come from Life and Biological Sciences --which is natural, however there is a lack of packages useful for social scientists in their own codes, language, protocols and kind of data. Here we propose to contribute closing this gap by providing an easy-to-use package that computes some of the most commonly used measures of diversity in social sciences --like the Blau Index, Shannon-Entropy, Hirschman Herfindahl Index--, but also allowing to take disparity and rareness of categories into account with measures like ubiquity or Rao-Stirling diversity.

%say something about all packages are related to biological sciences instead of social sciences 

The \CRANpkg{diveR} package allows calculating a large set of diversity used in different disciplines and explains them in simple way to promote interdisciplinary communication.

The \pkg{diveR} package allows researcher from any discipline to 
\begin{enumerate}
  \item import entity/categories matrices in different shapes (matrix and edge list), and
  \item calculate a large set of diversity measures. 
\end{enumerate}

%The next section explains first different diversity and similarity measures in a non-technical way. Subsequently we introduce the different functions of the package and their corresponding mathematical formula.


%This section may contain a figure such as Figure~\ref{figure:rlogo}.

\section{Install and requirements}
\pkg{diveR} package requieres R version 3.1.1 or further. 
Source code is available in \url{https://github.com/mguevara/diveR} and issues and requirements can be submitted via the \href{https://github.com/mguevara/diveR/issues}{git repository}.

\section{Example data}
As a example, we will use data from MIT's Pantheon project \citep{macroconnections_mit_medialab_pantheon_2014}. Pantheon allows to compare the diversity of occupations of the globally known persons of each country. For simplification reasons, we analyze the data of 10 countries. The full dataset can be accessed at \cite{yu_pantheon:_2015}.

In our data, the \dfn{entities} correspond to countries, and the \dfn{categories} are occupations. Note that other common name for entity is \emph{system} and for categories \emph{species}, we will use indistinctly both nominations.  
!!!the distinction between entity, object, system is unclear!!! Moreover, for the sake of simplicity we should actually stick to one name and not using synonyms. We can write that synonyms exist at the beginning and other appropriate places, but should then stick to using always the same term in the explanations!!!

Figure \ref{figure:pantheon} presents treemaps that visualise the data on global cultural production for three countries whose globally famous persons substantially differ with regard to their occupations. While the globally known persons of Latvia are mainly politicians, Uruguay's globally known people are mainly soccer players and Canada's most famous persons are working in Arts or Entertainment. South Africa's most famous persons are related to politics and sports. But besides the most important category of each country, we can also see in the treemaps the diversity of occupations of the famous persons of each country. We can see for example that Canada has a more diverse set of globally famous persons than Uruguay, Latvia and South Africa.

\begin{figure}[htbp]
  \centering
 \includegraphics[width=5.6cm]{imgPantheonUR} 
 \includegraphics[width=5.6cm]{imgPantheonLV} \\
   \includegraphics[width=5.6cm]{imgPantheonCA} 
      \includegraphics[width=5.6cm]{imgPantheonSA} 
  \caption{Globally known people for Uruguay, Latvia, Canada and South Africa. Size of the boxes is proportional (within the country) to the number of people assigned to each occupation and born in that country. Color according to main domains of occupations. Source \url{http://pantheon.media.mit.edu}}
  \label{figure:pantheon}
\end{figure}



\section{Input data}
%A problem with which many researchers --especially from social sciences-- struggle is different file and shape formats. We try to simplify this technical problem as far as possible by implementing different import and export options. In this section we present the possibility to import two different shapes of datasets:
The data objects to be processed by the functions of the \pkg{diveR} package, can be whether a matrix with values of abundance where the entities are the \code{rownames} and the categories are the \code{colnames}; or a data frame with three columns in this order: entity, category and value of abundance.

However to simplify the input of data from external files, we provide the function \code{read.data()}. This function reads CSV files and detects automatically if the data file is in one of these two shapes: matrix shape or edges shape, then it retrieves a data frame to be used by the functions of the package.

Here we detail both type of configurations for the external data files to be read.

\subsection{Matrix or table shape}
 Here, the rows represent the entities (e.g. countries) for which we want to measure diversity, and the columns, the categories (e.g. products, areas of science, species) which produces the diversity of the entities. We name this shape of the data as \emph{matrix} or \emph{table}. Each cell of the table provides a value of presence, intensity, abundance or production of the category in the entity. The Table \ref{tab:dataPantheon} shows an example of this type of data input where the rows are the entitites (countries), the columns are the categories (occupations) and the cells are numeric values of abundance in the sense of quantity of globally known people born in that country.
  
  \input{tables/dataPantheon.tex}
  
  The function \code{read.data()} returns the data reshaped as a data frame of three columns. That is, the \emph{panel} shape without \code{NA} values.
  
  The user must provide the path to the external CSV file by using the argument \code{path}. As an example we attached raw data files attached to the package. These files are accesible by using the function of the system \code{file()} which retrieves the local path to the CSV example files included with the package.
  
  \begin{example}
 path_to_file <- system.file("extdata", "PantheonMatrix.csv", package = "diveR")
 read.data(path = path_to_file)
 Using Country as id variables
         Country            variable value
1        Vietnam          Politician     6
2    New Zealand          Politician     2
3         Latvia          Politician     8
4        Uruguay          Politician     5
5          Chile          Politician    10
...
\end{example}

Note that as the original data comes in a matrix shape, the function \code{read.data()} is reshaping (\dfn{melting}) the matrix into a panel shape, that is why the message \code{Using Country as id variables} is printed. This functionality uses the package \CRANpkg{reshape2}.


  \subsection{Edges or panel shape}
  The second option of reading an external file, is the \emph{panel} or \emph{edges} shape. This way of representing the data is based in columns and should include the ordered columns: entity, category, and value. To read this type of file, the same function \code{read.data()} can be used, it detects automatically this type of configuration and retrieves the data frame needed. See Table \ref{tab:dataPantheonPanel} for an example of this type of external input.

\input{tables/dataPantheonPanel.tex}

Here an example, reading a data file with edges configuration.

\begin{example}
 path_to_file <- system.file("extdata", "PantheonEdges.csv", package = "diveR")
 read.data(path = path_to_file)
         Country          Occupation Export
1        Vietnam          Politician      6
2     New Zeland          Politician      2
3         Latvia          Politician      8
4        Uruguay          Politician      5
5          Chile          Politician     10
...
\end{example}


\subsection{Files in other formats}
To facilitate the import of data, we provide a parameter \code{type} that can be used to indicate if the data file is of type Stata or SPSS. This functionality depends on the package \CRANpkg{foreign}.

\begin{example}
#reading external Edges file
path_to_file <- system.file("extdata", "PantheonEdges.sps", package = "diveR")
pantheon <-  read.data(path = path_to_file, type='spss')
     
path_to_file <- system.file("extdata", "PantheonEdges.dta", package = "diveR")
pantheon <- read.data(path = path_to_file, type = 'stata')
\end{example}
 
\subsection{Data included}

Finally, to let the user experiment with the package and to conduct the examples in this article, we attached a data frame (\code{.RData}) in the edges shape that is the preferred input format for \pkg{diveR} package. The user can call this data frame by using the \code{data()} function. 

\begin{example}
  data(pantheon)
  pantheon
         Country          Occupation Export
1        Vietnam          Politician      6
2    New Zealand          Politician      2
3         Latvia          Politician      8
4        Uruguay          Politician      5
...
#characteristics of the example data
str(pantheon)
'data.frame':	119 obs. of  3 variables:
 $ Country   : Factor w/ 10 levels "Canada","Chile",..: 10 5 4 9 2 8 7 6 3 1 ...
 $ Occupation: Factor w/ 52 levels "Actor","Architect",..: 40 40 40 40 40 40 40 40 40 40 ...
 $ Export    : int  6 2 8 5 10 9 17 36 38 10 ...
\end{example}


For the rest of the document, we will use the variable \code{pantheon} as the data frame that contains the example data of Countries-Occupations and values of abudance.

%[In a next step we should also include the option of including nested matrices: e.g. Scimago areas and categories; or countries exports at 1,2,3,4 digit level. Therefore the categories need to be numerically classified in a hierarchical way, e.g. like SITC] 

%[Also we should put an example here. I think that scientific disciplines might be a good example, because then more disciplines can identify with it]

%\begin{example}
  %d <- readDiver(dataPantheon)
%\end{example}

\section{Computing the dimensions of diversity}
The two most intuitive dimensions of diversity are the number of categories present in a system, and the distribution of elements over the categories \citep{mcdonald_conceptualization_2003}. 
%A system that has a high number of species but this species are biased only to a few, not necessarily is quite diverse. In the same way, a system that   
\citet{stirling_general_2007} introduced a third dimension related to how different are the categories, this proxied by the distance between categories. 
These three dimensions are defined by Stirling as \dfn{variety}, \dfn{balance} and \dfn{disparity}. 
In this section we detail the functions to compute the measures associated to each dimension. We also provide a generic function that allows the user to compute measures that combine several dimensions and others that are more complex or parametrised.

\subsection{Variety}
Variety measures the number of species $N$ present in a system. This dimension measures \emph{how many types the entity has}. The function \code{variety}, returns a data frame with the values of variety and with the entities in the names of the rows. If the parameter \code{sort} is not set to false, the variety is ordered in decreasing order.


\begin{example}
data(pantheon)
variety(data = pantheon)
             variety
Canada            27
China             24
South Africa      16
Portugal          11
Latvia            10
New Zealand        9
Chile              7
Saudi Arabia       7
Uruguay            4
Vietnam            4

\end{example}

In our example, in Chile were born in total 26 globally known people while in Latvia only 18, however the variety of Latvia is greater than Chile's variety, 10 against 8 which allows us to conclude that Latvia is more diverse (considering only the dimension variety) than Chile if measured by global cultural production --as defined by Pantheon MIT \cite{macroconnections_mit_medialab_pantheon_2014}. In ecology, this measure is also called \dfn{richness} of the system.

\subsubsection{Ubiquity}
If the variety is considered from the point of view of the species, then we obtain a measure of the abundance/rareness of the specie. This measure in some cases is called \dfn{ubiquity} as in the case of the Economic Complexity \cite[p. 21]{hausmann_atlas_2011}. 
We include the measure of rareness of the species through the function \code{ubiquity()} that returns the number of systems where the species has presence in descending order from the most common to the less one.
In our example the occupation of politician, writer or soccer player is more common (ubiquitous) than the occupation of referee, sculptor or wrestler. 

\begin{example}
   ubiquity(data = pantheon)
                    ubiquity
Politician                10
Writer                     8
Soccer.Player              6
Singer                     5
Actor                      4
...
Referee                    1
Sculptor                   1
Wrestler                   1
\end{example}

\subsection{Balance}
Measures associated to the dimension of Balance are those that measure \emph{how much of each type the entity has}. These measures could be understood as statistical dispersion and are mainly a function of the proportion present of the category $i$ in the entity. While some of them measure the \dfn{evenness} of the distribution as the Entropy, other ones measure the \dfn{concentration}, as the Gini Index. The dimension balance is also known as \dfn{abundance}, mainly in ecological oriented sciences. Table \ref{tab:balance} summarises the main measures of balance included in \pkg{diveR} package.

The function \code{balance()} retrieves six of the most common measures to estimate the dimension balance of the diversity. 

\begin{example}
  balance(data = pantheon)
              entropy      gini   simpson berger-parker inverse-simpson  evenness
Canada       2.559899 0.8608372 0.1391628     0.3162393        7.185827 0.7767069
Chile        1.626709 0.7603550 0.2396450     0.3846154        4.172840 0.8359632
China        2.340469 0.8168554 0.1831446     0.3838384        5.460167 0.7364472
Latvia       1.889159 0.7654321 0.2345679     0.4444444        4.263158 0.8204514
New Zealand  2.106577 0.8673469 0.1326531     0.2142857        7.538462 0.9587447
Portugal     1.610050 0.7100340 0.2899660     0.4285714        3.448680 0.6714431
Saudi Arabia 1.331425 0.6546939 0.3453061     0.4857143        2.895981 0.6842170
South Africa 2.440072 0.8788927 0.1211073     0.2647059        8.257143 0.8800700
Uruguay      1.135551 0.6260388 0.3739612     0.5263158        2.674074 0.8191267
Vietnam      1.168518 0.6280992 0.3719008     0.5454545        2.688889 0.8429079
\end{example}

In our --toy-- example, it is interesting to note the case of New Zealand which Gini is high, that means, less concentrated than other countries. It is also comparable with Canada's Gini, even when variety is 27 for Canada and only 9 for New Zealand. Note that the measure Gini used here to measure concentration, is not the same measure widely spread to measure inequality, also called the Gini-index. 

\begin{table}[htdp]
\begin{center}
\begin{tabular}{l>{$}l<{$}l} 
  \toprule
  Measure & \text{Formula} & Reference \\
  \midrule
 Entropy & - \sum_i\left(p_i \log p_i\right) & \cite{shannon_mathematical_1948} \\
  Gini, Gini-Simpson,  Herfindahl–Hirschman index,  &  1 - \sum_i\left(p_i^2\right)  & \cite{gini_variabilita_1912,ceriani_origins_2011} \\ 
 Simpson & \sum_i\left(p_i^2\right)  &  \cite{simpson_measurement_1949} \\ 
 Eveness & -\sum\left(p_i \log p_i\right)/\log{N}  & \cite{pielou_introduction_1970}\\ 
 Inverse-Simpson &  \left(\sum\left(p_i^2\right)\right)^{-1} & \cite{simpson_measurement_1949} \\
   Berger-Parker & \max{\left(p_i\right)} & \cite{berger_diversity_1970} \\ 
 
 \bottomrule
\end{tabular}
\end{center}
\caption{Summary of measures computed by balance function. $p_i$ denotes the proportion of the $i$th category in the entity $j$. $N$ is the number of categories. $log$ is the natural logarithm.}
\label{tab:balance}
\end{table}

There are other balance measures that are not included in the function \code{balance()} that the user might want to use, such as !!!XYZ??? These other functions can be computed by using the function \code{diversity()} that is described in the Subsection \ref{sec:diversity}.

%TODO: more discussion about the balance measures, mainly Entropy is needed

\subsection{Disparity}
Any diversity indicator also depends upon the classification of species into different main categories; or in other words the disparity or dis-similarity between species. For example an artist and a singer are more similar than an chemist and a singer. Therefore, diversity measures are also closely related to distance and similarity measures like Euclidean distances, cosine similarities, Jaccard-Index or expert classifications of different categories. In consequence we implement in our package the option to calculate different distance matrices to measure the difference between species based in the input data. For this purpose we draw on the package \CRANpkg{proxy} which contains a variety of different measures. 

Before computing diversity, in order to let the user visualize the distance matrix to use in the calculation of disparity, we provide the function \code{distances()} that includes the argument 'method' that is the one that signes the method used to compute the distance, as Euclidean, Cosine or Jaccard. Note that even when Cosine is a measure of similarity, the proper transformation into a distance is performed by the \pkg{proxy} package. Cosine 'distance' is used as default method.

\begin{example}
 distances(pantheon)
                          Actor Architect    Artist  Astronaut Astronomer   Athlete
Actor               1.000000000 0.3535401 0.9057442 0.86670221 1.00000000 0.2648050
Architect           0.353540065 1.0000000 0.5917517 0.42264973 1.00000000 0.5917517
Artist              0.905744227 0.5917517 1.0000000 0.29289322 1.00000000 1.0000000
Astronaut           0.866702208 0.4226497 0.2928932 1.00000000 1.00000000 1.0000000
Astronomer          1.000000000 1.0000000 1.0000000 1.00000000 1.00000000 1.0000000
Athlete             0.264804973 0.5917517 1.0000000 1.00000000 1.00000000 1.0000000
...
\end{example}

Note that the distance matrix is a square matrix ($N \times N$) of categories, however for the disparity computation, only the upper triangle will be used, since the distance is considered symmetric, so the number of distances to be used is actually $\left(N^2 - N\right)/2$.
 
To compute the disparity, we provide the function \code{disparity()} that calls internally to the function \code{distance()} and measures the disparity according to the distance measure set by the parameter \code{method}.

%TODO: clean this function to include in diversity function as a especial binary case of RaoStirling diversity
\begin{example}
   disparity(pantheon, method='cosine')
\end{example}


In some cases, the user might also want to determine the distance between entities. This can be performed using the function \code{distances()} but applying the argument \code{agg\_type='col'}. Then a matrix of distances between systems is retrieved. Note again that further uses of this matrix should only consider the upper or lower triangular matrix.

\begin{example}
  distances(data = pantheon, agg_type = 'col')
                Canada     Chile      China     Latvia New Zealand  Portugal Saudi Arabia
Canada       1.0000000 0.7137852 0.62523975 0.73193144   0.2361383 0.7715579    0.8061653
Chile        0.7137852 1.0000000 0.17848785 0.14383139   0.5993832 0.1072028    0.3505866
China        0.6252398 0.1784878 1.00000000 0.07134351   0.4815593 0.2522419    0.2219146
Latvia       0.7319314 0.1438314 0.07134351 1.00000000   0.5275827 0.1834247    0.2359116
New Zealand  0.2361383 0.5993832 0.48155926 0.52758267   1.0000000 0.6618144    0.6757932
Portugal     0.7715579 0.1072028 0.25224186 0.18342466   0.6618144 1.0000000    0.3131248
Saudi Arabia 0.8061653 0.3505866 0.22191457 0.23591163   0.6757932 0.3131248    1.0000000
\end{example}

In our example, according to the data, countries like Canada and New Zealand seem to be closer. It is also the case of  Portugal and Chile.

\subsection{Diversity}\label{sec:diversity}
In order to provide to the user with the possibility of choosing her preferred measure, \pkg{diveR} provides the function \code{diversity()} which uses the parameter \code{type} to indicate the measure of diversity that the user wants to compute. This parameter accepts either the complete name of the diversity or just the initial, one letter or two letters, depending of the name. 

This function also allows to calculate measures that combine several dimensions of diversity, such as \dfn{rao} diversity or \dfn{rao-stirling} diversity. Also some parametric measures can be computed using this functions, as \dfn{true diversity} or \dfn{Rényi entropy}. Table XX shows the other measures included through this function.

If the user want to compute all the posible measures included in the package, the parameter \code{type} should be set to \code{all}.

\begin{example}
  diversity(data = pantheon, type = 'all')
    variety  entropy      gini   simpson true-diversity berger-parker inverse-simpson
Canada            27 2.559899 0.8608372 0.1391628             52     0.3162393        7.185827
Chile              7 1.626709 0.7603550 0.2396450             52     0.3846154        4.172840
China             24 2.340469 0.8168554 0.1831446             52     0.3838384        5.460167
Latvia            10 1.889159 0.7654321 0.2345679             52     0.4444444        4.263158
New Zealand        9 2.106577 0.8673469 0.1326531             52     0.2142857        7.538462
Portugal          11 1.610050 0.7100340 0.2899660             52     0.4285714        3.448680
Saudi Arabia       7 1.331425 0.6546939 0.3453061             52     0.4857143        2.895981
South Africa      16 2.440072 0.8788927 0.1211073             52     0.2647059        8.257143
Uruguay            4 1.135551 0.6260388 0.3739612             52     0.5263158        2.674074
Vietnam            4 1.168518 0.6280992 0.3719008             52     0.5454545        2.688889
             rényi-entropy  evenness rao-stirling
Canada           1.9721106 0.7767069     12.01241
Chile            1.4285967 0.8359632     14.91499
China            1.6974794 0.7364472     15.06282
Latvia           1.4500102 0.8204514     14.96385
New Zealand      2.0200181 0.9587447     12.94902
Portugal         1.2379917 0.6714431     14.54679
Saudi Arabia     1.0633239 0.6842170     14.40618
South Africa     2.1110786 0.8800700     15.07070
Uruguay          0.9836032 0.8191267     12.17966
Vietnam          0.9891281 0.8429079     14.55674
\end{example}

The user can also modify the parameters used by some of the diversity measures, as in the case of the Rao-Stirling diversity, where the user can define the parameters $\alpha$ or $\beta$ that are the exponents of the multiplication of the proportions of systems $i$ and $j$ and their correspondent distance (See table XX).

\begin{example}
  #if alpha = 0 and beta = 0 then RS diversity equals to disparity
  diversity(data=pantheon, type='rs', alpha = 1, beta=0)
               rao-stirling
Canada           11551.94
Chile            11551.94
China            11551.94
Latvia           11551.94
New Zealand      11551.94
...

  #if alpha = 0 and beta = 1 then RS diversity equals to balance
  diversity(data=pantheon, type='rs', alpha = 0, beta=1)
rao-stirling
Canada          0.4304186
Chile           0.3801775
China           0.4084277
Latvia          0.3827160
New Zealand     0.4336735
Portugal        0.3550170
Saudi Arabia    0.3273469
South Africa    0.4394464
Uruguay         0.3130194
Vietnam         0.3140496
\end{example}

It is also the case of the true diversity, where the parameter \code{q} can be set according to the user needs.

\section{Conclusion / Discussion}

...


%There cite \cite{stirling_general_2007} citep \citep{stirling_general_2007}, citet \citet{stirling_general_2007} will likely be several sections, perhaps including code snippets, such as:

%\section{Data export}
%Researcher form different fields work with different statistical programs and tools. Therefore it is important to provide a set of different simple export options for the results. This is especially important to facilitate interdisciplinary communication.

%\begin{example}
  %x <- 1:10
  %result <- myFunction(x)
%\end{example}


\bibliography{guevara-hartmann-mendoza}

\address{Miguel R. Guevara\\
  Universidad de Playa Ancha and Universidad Federico Santa María\\
  Valparaíso\\
  Chile\\}
\email{miguel.guevara@upla.cl}

\address{Dominik Hartmann\\
  Chair for Economics of Innovation, University of Hohenheim\\
  Wollgrasweg 23, 70599, Stuttgart\\
  Germany\\}
\email{hartmado@mit.edu}

\address{Marcelo Mendoza\\
  Universidad Federico Santa María\\
  Av. Vicuna Mackenna 3939, San Joaquin, Santiago\\
  Chile\\}
\email{mmendoza@inf.utfsm.cl}


